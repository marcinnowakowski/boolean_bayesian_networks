The primary goal of the data generation phase was to populate the hyperparameter space defined in the previous sections. 
Due to the constraints imposed by the attractor ratio stochastic nature, not every combination of parameters resulted in a successful file generation. 
In total, the generation process yielded \textbf{39,880} successful trajectory files across all considered networks. 

The distribution of successfully generated datasets per network dimension is provided in Table \ref{tab:dimension_counts}.

\begin{table}[hbtp]
\centering
\begin{tabular}{lc}
\hline
\textbf{Dimension} & \textbf{Successful Files} \\ \hline
5d  & 6,549 \\
6d  & 4,291 \\
7d  & 7,763 \\
8d  & 7,511 \\
9d  & 8,825 \\
10d & 4,941 \\ \hline
\textbf{Total} & \textbf{39,880} \\ \hline
\end{tabular}
\caption{Count of successfully generated datasets for different dimensionality.}
\label{tab:dimension_counts}
\end{table}

The overall success rate relative to the target attractor ratio is visualized in Figure \ref{fig:hist_success}. 
As can be seen, for most generated files, the trajectories were almost always in the attractor states.
This fact rises primarily because for large trajectories, it is very likely to fall into the attractor basin early and remain there for the rest of the generated trajectory.

\begin{figure}[hbtp]
    \centering
    \includegraphics[width=0.8\textwidth]{a03_results/histogram_success_ratio.png}
    \caption{Distribution of successfully generated files in the range of target attractor ratios.}
    \label{fig:hist_success}
\end{figure}