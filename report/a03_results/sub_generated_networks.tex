% Generated Networks

For the benchmark study, a total of 50 synthetic Boolean networks were generated using the custom generator described in Section 2.1. The networks span 10 different dimensions, with 5 networks generated for each dimension.

\subsubsection{Network Dimensions}

The generated networks selected for analysis cover the following dimensions:
\begin{table}[h!]
    \centering
    \label{tab:network_specs}
    \begin{tabular}{ccc}
        \hline
        \textbf{Dimension} & \textbf{Number of Networks} & \textbf{States per Network ($2^n$)} \\ \hline
        5D  & 5 & 32      \\
        6D  & 3 & 64      \\
        7D  & 5 & 128     \\
        8D  & 3 & 256     \\
        9D  & 3 & 512     \\
        10D & 3 & 1,024   \\ \hline
    \end{tabular}
\end{table}

\subsubsection{Network Properties}

All networks share the following characteristics:
\begin{itemize}
    \item \textbf{Fixed dependency limit:} Each variable's update function depends on exactly 3 other variables.
    \item \textbf{Reproducibility:} Each network was generated with a unique random seed, ensuring exact reproducibility.
    \item \textbf{Boolean functions:} Update functions are represented in Sum-of-Products (SOP) form with 2--6 true outputs per function.
\end{itemize}

\begin{table}[h!]
    \centering
    \caption{Structural properties of generated networks}
    \label{tab:network_structure}
    \begin{tabular}{ccccccc}
        \hline
        \textbf{Dim} & \textbf{Seed} & \textbf{Parentless} & \textbf{Fixed Pts} & \textbf{Attractors} & \textbf{Trans. SCCs} & \textbf{SCC Edges} \\ \hline
        5  & 1   & 2 & 2 & 2 & 30  & 80   \\
        5  & 2   & 1 & 1 & 1 & 31  & 80   \\
        5  & 3   & 1 & 1 & 1 & 31  & 80   \\
        5  & 4   & 1 & 1 & 1 & 31  & 80   \\
        5  & 5   & 1 & 2 & 2 & 30  & 80   \\
        6  & 101 & 0 & 2 & 3 & 20  & 57   \\
        6  & 202 & 0 & 0 & 1 & 0   & 0    \\
        6  & 303 & 3 & 0 & 1 & 44  & 120  \\
        7  & 3   & 0 & 2 & 3 & 9   & 22   \\
        7  & 26  & 2 & 1 & 3 & 5   & 10   \\
        7  & 35  & 2 & 2 & 3 & 62  & 192  \\
        7  & 40  & 3 & 4 & 4 & 79  & 252  \\
        7  & 43  & 4 & 1 & 3 & 67  & 195  \\
        8  & 101 & 0 & 1 & 1 & 2   & 2    \\
        8  & 202 & 4 & 0 & 2 & 39  & 135  \\
        8  & 303 & 0 & 0 & 2 & 7   & 16   \\
        9  & 101 & 0 & 1 & 1 & 16  & 48   \\
        9  & 202 & 3 & 2 & 2 & 72  & 222  \\
        9  & 303 & 0 & 0 & 1 & 1   & 1    \\
        10 & 101 & 0 & 0 & 1 & 23  & 96   \\
        10 & 202 & 1 & 3 & 4 & 163 & 697  \\
        10 & 303 & 1 & 0 & 2 & 219 & 1156 \\ \hline
    \end{tabular}
\end{table}

\subsubsection{Attractor and SCC Size Distribution}

Table~\ref{tab:scc_sizes} shows the sizes of attractors and largest transient SCCs for each network. Lower-dimensional networks (5D) exhibit only fixed-point attractors of size 1, while higher-dimensional networks show more diverse attractor structures, including large cyclic attractors (e.g., 956 states in 10D seed 101). The transient SCC sizes reveal that higher-dimensional networks often contain one or two dominant transient components capturing most of the state space flow.

\newpage

\begin{table}[h!]
    \centering
    \caption{Attractor and transient SCC sizes}
    \label{tab:scc_sizes}
    \begin{tabular}{ccll}
        \hline
        \textbf{Dim} & \textbf{Seed} & \textbf{Attractor Sizes} & \textbf{Transient SCC Sizes (top 5)} \\ \hline
        5  & 1   & [1, 1]           & [1, 1, 1, 1, 1]        \\
        5  & 2   & [1]              & [1, 1, 1, 1, 1]        \\
        5  & 3   & [1]              & [1, 1, 1, 1, 1]        \\
        5  & 4   & [1]              & [1, 1, 1, 1, 1]        \\
        5  & 5   & [1, 1]           & [1, 1, 1, 1, 1]        \\
        6  & 101 & [32, 1, 1]       & [4, 4, 2, 2, 2]        \\
        6  & 202 & [64]             & --                     \\
        6  & 303 & [2]              & [16, 2, 2, 2, 1]       \\
        7  & 3   & [8, 1, 1]        & [110, 1, 1, 1, 1]      \\
        7  & 26  & [2, 2, 1]        & [62, 58, 1, 1, 1]      \\
        7  & 35  & [64, 1, 1]       & [1, 1, 1, 1, 1]        \\
        7  & 40  & [1, 1, 1, 1]     & [38, 2, 2, 2, 2]       \\
        7  & 43  & [4, 4, 1]        & [8, 8, 6, 6, 4]        \\
        8  & 101 & [1]              & [253, 2]               \\
        8  & 202 & [2, 2]           & [182, 8, 8, 8, 2]      \\
        8  & 303 & [2, 2]           & [244, 2, 2, 1, 1]      \\
        9  & 101 & [1]              & [496, 1, 1, 1, 1]      \\
        9  & 202 & [1, 1]           & [230, 210, 1, 1, 1]    \\
        9  & 303 & [2]              & [510]                  \\
        10 & 101 & [956]            & [32, 4, 4, 4, 4]       \\
        10 & 202 & [2, 1, 1, 1]     & [326, 229, 48, 28, 24] \\
        10 & 303 & [512, 2]         & [24, 24, 16, 16, 8]    \\ \hline
    \end{tabular}
\end{table}

\subsubsection{Rationale for Dimension Selection}

The dimension range from 5 to 16 was chosen to:
\begin{enumerate}
    \item \textbf{Span tractable to challenging cases:} Lower-dimensional networks (5--7D) provide scenarios where reconstruction should be feasible, while higher-dimensional networks (13--16D) test the limits of the reconstruction algorithm.
    \item \textbf{Enable scaling analysis:} The exponential growth of state space ($2^n$) across dimensions allows for systematic evaluation of how reconstruction quality degrades with increasing network complexity.
    \item \textbf{Match biological relevance:} Many real biological signaling pathways involve 5--15 key regulatory components, making this range practically relevant.
\end{enumerate}
