% Model Selection Pipeline

The biological validation phase required selecting real Boolean network models from established repositories. We developed a multi-stage selection pipeline to identify suitable candidates from the biodivine-boolean-models repository, a comprehensive collection of published biological Boolean network models.

\subsection{Data Source}

The biodivine-boolean-models repository contains over 200 curated Boolean network models from the literature, representing various biological systems including cell cycle regulation, signaling pathways, and gene regulatory networks. Models are provided in the \texttt{.bnet} format, specifying Boolean update functions for each network variable.

\subsection{Selection Pipeline}

The selection process was implemented through a series of Python scripts that form the \texttt{biodivine\_importer} module:

\subsubsection{Stage 1: Import and Organization}

\begin{enumerate}
    \item \textbf{Model Import (\texttt{importer.py}):} All \texttt{model.bnet} files were extracted from the source repository and renamed to a standardized format (\texttt{model\_XXX.bnet}) based on their unique identifiers.
    \item \textbf{Dimension-based Organization (\texttt{organize\_by\_dimension.py}):} Models were organized into subdirectories based on their dimensionality (number of variables). This yielded models ranging from 4D to over 800D, with the majority concentrated in the 5--50 variable range.
\end{enumerate}

\subsubsection{Stage 2: Compatibility Filtering}

\begin{enumerate}
    \item \textbf{External Input Detection (\texttt{count\_inputs.py}):} Many biological models include external input variables (environmental signals, experimental conditions) that are referenced but not defined within the network. Models with external inputs require special handling---either the inputs must be substituted with constant values, or multiple network variants must be generated for different input combinations.
    \item \textbf{Dimension Filtering:} Based on the results from Part I, only models with dimensions between 4 and 16 were retained, as higher-dimensional networks exhibited poor reconstruction quality regardless of parameter tuning.
\end{enumerate}

\subsubsection{Stage 3: Format Conversion}

The \texttt{convert\_to\_functions.py} script performed the following transformations:
\begin{itemize}
    \item \textbf{Variable Normalization:} Original variable names (e.g., \texttt{v\_CcrM}, \texttt{v\_CtrA}) were mapped to a standardized format (\texttt{x1}, \texttt{x2}, ...) while preserving the original mapping in documentation.
    \item \textbf{External Input Substitution:} For models with external inputs, multiple variants were generated by substituting all possible combinations of constant values (0 or 1) for each external input.
    \item \textbf{Transition Graph Generation:} Asynchronous state transitions were computed from the Boolean functions, consistent with the synthetic network format.
\end{itemize}

\subsection{Final Model Selection}

From the filtered and converted models, three networks were selected for the biological validation experiments:

\begin{enumerate}
    \item \textbf{bn\_109\_5d.py (5D):} A model of the \textit{Caulobacter crescentus} cell cycle regulatory network, featuring 5 key regulators: CcrM, CtrA, DnaA, GcrA, and SciP. This compact model has no external inputs and represents a well-studied bacterial cell cycle.
    
    \item \textbf{bn\_158\_7d.py (7D):} A model of the bacteriophage $\lambda$ lysis-lysogeny decision circuit, with 7 variables including CI, Cro, and N proteins. This classic system demonstrates bistable genetic switching behavior.
    
    \item \textbf{bn\_063\_10d\_011.py (10D):} A model of embryonic development regulatory network with 10 variables. This variant was generated by substituting external inputs (\texttt{v\_Ge}=0, \texttt{v\_Le}=1, \texttt{v\_Lem}=1) with specific constant values, representing a particular developmental context.
\end{enumerate}

\subsection{Selection Rationale}

The three selected models were chosen to:
\begin{itemize}
    \item \textbf{Span the tractable dimension range:} 5D, 7D, and 10D represent low, medium, and borderline-high complexity based on Part I results.
    \item \textbf{Include biological diversity:} The models represent different biological contexts (bacterial cell cycle, phage decision, embryonic development).
    \item \textbf{Minimize external dependencies:} Priority was given to models with few or no external inputs to ensure self-contained network dynamics.
\end{itemize}

\subsection{Structural Properties of Selected Models}

Table~\ref{tab:biodivine_structure} summarizes the structural properties of the three selected biological networks. Table~\ref{tab:biodivine_scc_sizes} shows the attractor and transient SCC size distributions for each model.

\begin{figure}[hbtp]
    \centering
    \includegraphics[width=0.6\textwidth]{b02_model_selection/bn_109_5d_scc_simplified.png}
    \caption{SCC graph of the Caulobacter cell cycle network (5D). The single fixed-point attractor is shown in green, with 7 transient SCCs forming the basin of attraction. The largest transient SCC contains 25 states.}
    \label{fig:scc_5d}
\end{figure}

\begin{figure}[hbtp]
    \centering
    \includegraphics[width=0.8\textwidth]{b02_model_selection/bn_158_7d_scc_simplified.png}
    \caption{Simplified SCC graph of the phage lambda decision network (7D). Two attractors are visible: a fixed point and a 2-state cycle representing the bistable lysis-lysogeny switch. The 79 transient SCCs reflect the complex decision-making landscape.}
    \label{fig:scc_7d}
\end{figure}

\begin{figure}[hbtp]
    \centering
    \includegraphics[width=0.8\textwidth]{b02_model_selection/bn_063_10d_011_scc_simplified.png}
    \caption{Simplified SCC graph of the embryonic development network (10D). Despite having 704 transient SCCs, the network converges to a single fixed-point attractor. The two largest transient SCCs contain 221 and 100 states respectively.}
    \label{fig:scc_10d}
\end{figure}

\begin{table}[h!]
    \centering
    \caption{Structural properties of selected biological networks}
    \label{tab:biodivine_structure}
    \begin{tabular}{clccccc}
        \hline
        \textbf{Dim} & \textbf{Parentless} & \textbf{Fixed Pts} & \textbf{Attractors} & \textbf{Trans. SCCs} & \textbf{SCC Edges} \\ \hline
        5D   & 1 & 1 & 1 & 7   & 13   \\
        7D     & 1 & 1 & 2 & 79  & 294  \\
        10D     & 1 & 1 & 1 & 704 & 3723 \\ \hline
    \end{tabular}
\end{table}

\begin{table}[h!]
    \centering
    \caption{Attractor and transient SCC sizes for biological networks}
    \label{tab:biodivine_scc_sizes}
    \begin{tabular}{clll}
        \hline
        \textbf{Dim} & \textbf{Description} & \textbf{Attractor Sizes} & \textbf{Transient SCC Sizes (top 5)} \\ \hline
        5  & Caulobacter cell cycle   & [1]      & [25, 1, 1, 1, 1]      \\
        7  & Phage lambda decision    & [2, 1]   & [31, 11, 4, 2, 2]     \\
        10 & Embryonic development    & [1]      & [221, 100, 1, 1, 1]   \\ \hline
    \end{tabular}
\end{table}

All three networks exhibit fixed-point attractors, with the phage lambda model additionally containing a 2-state cyclic attractor representing the bistable lysis-lysogeny switch. The 10D embryonic development network has a notably large number of transient SCCs (704), reflecting the complex developmental trajectory landscape.

\subsection{SCC Graph Visualizations}

To better understand the structure of the selected networks, we generated condensation graphs showing the relationships between strongly connected components. In these visualizations, green nodes represent attractors (terminal SCCs), blue nodes represent transient SCCs, and node labels indicate the number of states in each SCC.
