\documentclass{article}
\usepackage{graphicx} % Required for inserting images
\usepackage[colorinlistoftodos, prependcaption, textsize=tiny]{todonotes}

\title{SAD project}
\author{Mateusz Wawrzyniak}
\date{January 2026}

\begin{document}

\maketitle

\begin{center}
    \huge \textbf{PART I: Benchmark on Synthetic Boolean Networks} 
\end{center}

\section{Introduction}

\section{experimental setup}

\subsection{network generation}
\todo{Marcin}
czym bylo zaimplementowane to generowanie, jakie mocne/slabe strony tego narzedzia.
nie mowimy jeszcze o tym jakie zostaly wygenerowane, tylko o samej metodologii generowania

\subsection{dataset generation}
\todo{Mateusz}

\subsubsection{parametrization}
jakie parametry byly wybrany, jakie wybrane hard-limity na przestrzen parametrow, jakie mody etc.

\subsubsection{preprocessing of the networks}
kompilacja do state-networks, identyfikacja stanów nalezacych do atraktorow

\subsubsection{creating the trajectories}
optymalizacje produkowania trajektorii, pooling, multithreading

\subsubsection{running bnfinder}
jak to dzialalo, wybrany output i dlaczego

\subsection{network reconstruction and evaluation}
\todo{Maciej}

\subsubsection{reconstructing the networks}
w jaki sposob z outputu BNFa powstawala siec booelowska zrekonstuowana. dlaczego tak, wady i zalety

\subsubsection{developed reconstruction quality metrics}
wybrane metryki do ewaluacji, dlaczego, wady i zalety

\section{First iteration}

\subsection{generated networks - first iteration}
\todo{Marcin}
krotko, jakie sieci zostaly wybrane, ile ich bylo

\subsection{datasets generated - first iteration}
\todo{Mateusz}
jaka przestrzen parametrow wybrana, dlaeczego taka, czemu 13d i wyzej nie podlegalo dalszej analizie

\subsection{results - first iteration}
\todo{Maciej}
czego nauczylismy sie z pierwszej iteracji rekonstrukcji

\section{Second iteration}

\subsection{generated networks - second iteration}
\todo{Marcin}
krotko, jakie sieci zostaly wybrane, ile ich bylo. Co bylo, czego nie bylo w pierwszej iteracji i dlaczego

\subsection{datasets generated - second iteration}
\todo{Mateusz}
jaka przestrzen parametrow wybrana. Co korygowala wzgledem pierwszej iteracji, co nam pozwalala

\subsection{results - second iteration}
\todo{Maciej}
czego sie dowiedzielismy, co uscislilismy w porownaniu do pierwszej iteracji

\section{conclusions.}
\todo{Wszyscy}
przede wszystkim nasza konkluzja co do guideline'u jakie parametry powinno sie stosowac w zaleznosci od specyfiki naszej sieci

\begin{center}
    \huge \textbf{PART II: Biological Model Validation} 
\end{center}

\section{introduction}
\todo{Marcin}
co to za model, jaka jest jego specyfika, moze kilka slow co on biologicznie znaczy i jaka jest jego wartosc

\section{methodology}
\todo{Marcin}
jakie parametry trajektorii/rekonstrukcji wybrano i dlaczego takie, w odniesieniu do konkluzji czesci pierwszej

\section{results}
\todo{Wszyscy}
jak dobrze poszla rekonstrukcja, czy zgodnie z przewidywaniami, czy podobnie jak wychodzilo w sieciach syntetycznych

\section{conclusions / closing remarks}
\todo{Marcin}
krotko, jak sobie poradzil pipeline w empirycznej sieci. moze krotko co oznaczalaby zrekonstruowana siec biologicznie np. wg zrekonstruowanej sieci jak dzialaby szlak bialkowy etc.

\end{document}
