\documentclass{article}
\usepackage{graphicx} % Required for inserting images
\usepackage[colorinlistoftodos, prependcaption, textsize=tiny]{todonotes}

\title{SAD project}
\author{}
\date{January 2026}

\begin{document}

\maketitle

\begin{center}
    \huge \textbf{Part I: BNFinder2 benchmark using synthetic Boolean Networks} 
\end{center}

\section{Introduction}

\section{experimental setup}

\subsection{Network Generation}
% Network Generation Methodology

For generating synthetic Boolean networks, we developed a custom Python-based generator with a fixed dependency limit. The generator accepts the network dimension $n$ (number of variables) as an input parameter and produces networks where each variable's update function depends on exactly 3 other variables.

\subsubsection{Generation Algorithm}

The generation process consists of two main phases:

\paragraph{Phase 1: Function Definition}
For each variable $x_i$ in the network:
\begin{enumerate}
    \item \textbf{Dependency Selection:} A set of exactly 3 dependency variables is randomly selected from all network variables (potentially including $x_i$ itself).
    \item \textbf{Truth Table Construction:} A Boolean function $f_i$ is defined over the 3 dependency variables. The function is represented as a truth table with $2^3 = 8$ entries. To ensure non-trivial dynamics, the number of ``true'' outputs is constrained between 2 and 6 out of 8 possible input combinations.
\end{enumerate}

\paragraph{Phase 2: Transition Computation}
The asynchronous state transitions are derived directly from the update functions. For each state $s$, a transition to state $s'$ (differing only in bit $i$) is possible if and only if $f_i(s) \neq s_i$. States with no possible transitions become fixed points (self-loops).

\subsubsection{Output Format}

The generator produces:
\begin{itemize}
    \item \textbf{Boolean expressions:} Each update function is converted to Sum-of-Products (SOP) form, facilitating subsequent analysis and reconstruction evaluation.
    \item \textbf{Transition graph:} A complete mapping from each state to its possible successor states under asynchronous update semantics.
\end{itemize}

\subsubsection{Network Analysis}

The generator includes a built-in \texttt{analyze\_network} function that computes structural properties of the generated transition graph:
\begin{itemize}
    \item \textbf{State space metrics:} Total number of states ($2^n$) and transitions in the graph.
    \item \textbf{Parentless states:} States with no incoming transitions (initial states reachable only from themselves).
    \item \textbf{Fixed points:} States that are self-loops, i.e., where no variable can change.
    \item \textbf{Strongly Connected Components (SCCs):} Identified using Tarjan's algorithm to find maximal sets of mutually reachable states.
    \item \textbf{Attractors:} Terminal SCCs from which no external transitions exist, representing the long-term behavior of the network.
\end{itemize}

\subsubsection{Strengths}

\begin{itemize}
    \item \textbf{Controlled complexity:} The hard limit on input dependencies (at most 3) ensures that networks remain tractable for Bayesian network reconstruction, which might struggle with high-arity functions.
    \item \textbf{Reproducibility:} Configurable random seed enables exact reproduction of generated networks.
    \item \textbf{Flexible parameterization:} Network dimension $n$ and function complexity can be adjusted via configuration parameters.
    \item \textbf{Built-in analysis:} The \texttt{analyze\_network} function provides comprehensive structural insights without requiring external tools.
\end{itemize}

\subsubsection{Limitations}

\begin{itemize}
    \item \textbf{Uniform dependency count:} All variables have at most 3 dependencies; real biological networks may exhibit more heterogeneous connectivity patterns.
    \item \textbf{Random structure:} The generator does not impose specific topological constraints (e.g., scale-free or small-world properties) that are common in biological networks.
    \item \textbf{No biological priors:} Functions are purely random within the specified constraints, without incorporating biologically motivated motifs or regulatory patterns.
    \item \textbf{Asynchronous semantics only:} While the generator natively produces asynchronous transitions, synchronous dynamics must be derived separately.
\end{itemize}


\subsection{Dataset Generation}
\subsubsection{Parametrization}
To evaluate the reconstruction capabilities of the BNFinder2 algorithm, a 4-dimensional hyperparameter space was defined to describe the characteristics of the generated trajectories. 
The analysis focuses on two primary update modes, \textbf{synchronous}, characterized by the simultaneous update of all nodes, and \textbf{asynchronous}, where a single randomly selected node is updated per time step. 

The following  hyperparameters were varied to generate the simulated datasets:
\begin{itemize}
    \item \textbf{Sampling frequency ($f$):} The step interval between recorded states.
    \item \textbf{Dataset size ($s$):} The number of independent trajectories included in a single input file for BNFinder2.
    \item \textbf{Trajectory length ($l$):} The number of steps recorded per trajectory.
    \item \textbf{Attractor ratio ($at$):} The target proportion of states belonging to attractors versus transient states in each trajectory.
\end{itemize}

An initial sparse parameter search was performed to find more precise extent and boundaries of these parameters. 
This initial phase explored the following ranges:
\begin{itemize}
    \item \textbf{Network dimensions ($n$):} 5, 7, 10, 13, 16.
    \item \textbf{Sampling frequencies ($f$):} 1, 5, 10.
    \item \textbf{Trajectory lengths ($l$):} 10, 50, 100, 500.
    \item \textbf{Dataset sizes ($s$):} 1, 10, 100, 300.
    \item \textbf{Attractor ratio ($at$):} $0.1 \times n \pm 0.05$ for $n = \{1, \dots, 10\}$.
\end{itemize}

Following this initial spares search, a visual analysis of the marginal distributions of the reproduction quality (QC) metrics was performed.
By summing across marginal distributions e.g. aggregating results across all lengths and sizes for a specific frequency, we were able to make an educated guess regarding the actual parameter space required for sensible exploration.

The observations from the sparse search led to the following methodological adjustments:
\begin{enumerate}
    \item \textbf{Dimensionality constraints:} 
    It was observed that for any network with dimensionality $n \geq 10$, the reproduction quality metrics were significantly low.
    Furthermore, there was no significant difference in performance for networks exceeding 10 dimensions.
    The algorithm simply performed poorly regardless of the parameter tuning.
    Consequently, we chose to focus on network dimensions $n \in [5, 6, 7, 8, 9, 10]$.
    \item \textbf{Sampling frequency:} 
    Non-trivial relationships were found between the sampling frequency and the quality metrics.
    To capture these dynamics, the sampling space was expanded to: $f \in \{1, 2, 3, 5, 7, 10\}$.
    \item \textbf{Resolution refinement for lengths and sizes:} 
    Non-trivial QC metric values were obtained across the initial spans for trajectory lengths and dataset sizes.
    To improve the resolution of the analysis, a log-like progression was introduced for these parameters:
    \begin{itemize}
        \item \textbf{Lengths ($l$):} $\{10, 20, 50, 100, 150, 200, 300, 400\}$.
        \item \textbf{Sizes ($s$):} $\{1, 10, 20, 50, 100, 150, 250\}$.
    \end{itemize}
    \item \textbf{Attraction ratio control:} 
    The attractor ratio is the only parameter that could not be directly controlled through initialization.
    Instead, it was controlled through a brute-force pooling approach, as explained in the following subsection.
    The span and resolution with increments of 0.1 from the initial generation were maintained for the final study.
\end{enumerate}

\subsubsection{Preprocessing of the networks}
The preprocessing pipeline transformed raw Boolean functions provided in the network-generation module into a format more suited for computationally-dense simulation:
\begin{enumerate}
    \item \textbf{Compilation:} 
    Logical transition functions provided as raw strings, were cleaned and their operators ($\sim, \&, |$) were mapped to Python logical operators .
    These strings were then compiled and stored in a dictionary keyed by node names. 
    This transformation was a necessity to easily and efficiently handle the Boolean network representations during simulation steps.

    \item \textbf{State Transition System (STS) generation:}
    A  graph of state transitions was constructed using the dictionary of compiled code objects. 
    In synchronous mode, the system updates all nodes simultaneously resulting in each state pointing to exactly one successor. 
    In asynchronous mode, the system updates only one node at a time. 
    This explicit construction was done in order to generate the trajectories of the networks more efficiently.
    Traversing a pre-computed state transition graph is computationally much faster than repeatedly inferring the next step from raw Boolean logic.
    
    \item \textbf{Attractor identification:} 
    Tarjan's algorithm was employed to find Strongly Connected Components (SCC) within the STS.
    This is a common and highly efficient O(N) algorithm used in graph theory to identify sets of nodes where every node is reachable from every other node in the same set.
    Attractors were identified as terminal SCCs, which are components from which no external transitions to other states exist.
    Identifying the attractors prior to trajectory generation was necessary to easily control the the ratio of the trajectory spent in attractor states versus transient states.
\end{enumerate}

\subsubsection{Generating the trajectories}

A single trajectory was created by traversing either an asynchronous or synchronous state transition graph for a specified length.
After the simulation, the trajectory was downsampled to the required sampling frequency. 
At this point, we had a single trajectory with a specific mode, frequency, and length. 
In theory, we could repeat this deterministic process as many times as needed to reach the target SIZE ($s$).

The main problem was the attractor ratio, as it is a stochastic characteristic of a trajectory.
We could not pre-specify this value before starting the simulation.
Simply brute-forcing the generation until we obtained a specific range of attractor ratios could take a very long time.
For example, producing 250 trajectories that all have a length of 400 in a 5D network while hoping to get an attractor ratio of $0.1 \pm 0.05$ for every single one is nearly impossible through simple trial and error.

Because of this, a \textbf{pooling strategy} was implemented:
\begin{itemize}
    \item For every pair of sampling frequency ($f$) and trajectory length ($l$) a large \texttt{trajectory\_pool} of 10,000 trajectories is pre-simulated.
    \item Each trajectory generated within the pool is classified by its actual, calculated attractor ratio.
    \item The final datasets are created by sampling $s$ trajectories from the pool that satisfy the specific condition: $|actual\_ratio - target\_ratio| \leq \epsilon$.
\end{itemize}

To make the process faster, the trajectory generation was parallelized using the \texttt{multiprocessing} module.
Tasks for generating the pools and selecting the final trajectories were distributed across all available CPU cores using \texttt{multiprocessing.Pool.map}. 
This approach significantly reduced the processing time, which was especially important for high-dimensional networks where state spaces are much larger.

\subsubsection{Running BNFinder}
The reconstruction process was automated with a Python script that interfaced with the BNFinder2 tool within a dedicated Conda environment.


\textbf{Isolation:} 
The \texttt{subprocess} module was used to execute the \texttt{bnf} command with specific flags.
We did it for the sake of completely isolating the Python 2 environment required by BNFinder2 from the Python 3 main script used for data generation.

\textbf{Settings:} 
In the initial trajectory generation, both MDL and BDe scoring criteria were used in BNFinder2.
After the visual inspection of the QC metric distributions, we noticed that there were almost no cases in which BDe scoring produced better reconstructions than MDL scoring and so, for the final analysis, we used only MDL scoring.
More support with the appropriate distributions will be shown later.
A parent limit of 3 was used, as stated in the assignment.
The \texttt{-g} flag was enabled to allow for self-loops. 
This is a requirement for Dynamic Bayesian Networks (DBN), where the state of a node at $t+1$ may depend on its own state at $t$, particularly in asynchronous datasets where nodes may remain unchanged between steps.

\textbf{Output:} 
Results were saved in the BIF (Bayesian Interchange Format), a standardized structure for the evaluation against the original network topologies.
We found this format to be the most informative as it encodes the Conditional Probability Tables (CPTs), defining the probability of each node's state based on the states of its parents. 

\subsection{network reconstruction and evaluation}
\todo{Maciej}

\subsubsection{reconstructing the networks}
w jaki sposob z outputu BNFa powstawala siec booelowska zrekonstuowana. dlaczego tak, wady i zalety

\subsubsection{developed reconstruction quality metrics}
wybrane metryki do ewaluacji, dlaczego, wady i zalety

\section{Results}

\subsection{Generated Networks}
% Generated Networks

For the benchmark study, a total of 50 synthetic Boolean networks were generated using the custom generator described in Section 2.1. The networks span 10 different dimensions, with 5 networks generated for each dimension.

\subsubsection{Network Dimensions}

The generated networks cover the following dimensions:
\begin{itemize}
    \item \textbf{5D networks:} 5 networks with 32 states each
    \item \textbf{6D networks:} 5 networks with 64 states each
    \item \textbf{7D networks:} 5 networks with 128 states each
    \item \textbf{8D networks:} 5 networks with 256 states each
    \item \textbf{9D networks:} 5 networks with 512 states each
    \item \textbf{10D networks:} 5 networks with 1,024 states each
    \item \textbf{11D networks:} 5 networks with 2,048 states each
    \item \textbf{12D networks:} 5 networks with 4,096 states each
    \item \textbf{13D networks:} 5 networks with 8,192 states each
    \item \textbf{16D networks:} 5 networks with 65,536 states each
\end{itemize}

\subsubsection{Network Properties}

All networks share the following characteristics:
\begin{itemize}
    \item \textbf{Fixed dependency limit:} Each variable's update function depends on exactly 3 other variables.
    \item \textbf{Reproducibility:} Each network was generated with a unique random seed, ensuring exact reproducibility.
    \item \textbf{Boolean functions:} Update functions are represented in Sum-of-Products (SOP) form with 2--6 true outputs per function.
\end{itemize}

\subsubsection{Rationale for Dimension Selection}

The dimension range from 5 to 16 was chosen to:
\begin{enumerate}
    \item \textbf{Span tractable to challenging cases:} Lower-dimensional networks (5--7D) provide scenarios where reconstruction should be feasible, while higher-dimensional networks (13--16D) test the limits of the reconstruction algorithm.
    \item \textbf{Enable scaling analysis:} The exponential growth of state space ($2^n$) across dimensions allows for systematic evaluation of how reconstruction quality degrades with increasing network complexity.
    \item \textbf{Match biological relevance:} Many real biological signaling pathways involve 5--15 key regulatory components, making this range practically relevant.
\end{enumerate}


\subsection{Generated datasets}
The primary goal of the data generation phase was to populate the hyperparameter space defined in the previous sections. 
Due to the constraints imposed by the attractor ratio stochastic nature, not every combination of parameters resulted in a successful file generation. 
In total, the generation process yielded \textbf{39,880} successful trajectory files across all considered networks. 

The distribution of successfully generated datasets per network dimension is provided in Table \ref{tab:dimension_counts}.

\begin{table}[hbtp]
\centering
\begin{tabular}{lc}
\hline
\textbf{Dimension} & \textbf{Successful Files} \\ \hline
5d  & 6,549 \\
6d  & 4,291 \\
7d  & 7,763 \\
8d  & 7,511 \\
9d  & 8,825 \\
10d & 4,941 \\ \hline
\textbf{Total} & \textbf{39,880} \\ \hline
\end{tabular}
\caption{Count of successfully generated datasets for different dimensionality.}
\label{tab:dimension_counts}
\end{table}

The overall success rate relative to the target attractor ratio is visualized in Figure \ref{fig:hist_success}. 
As can be seen, for most generated files, the trajectories were almost always in the attractor states.
This fact rises primarily because for large trajectories, it is very likely to fall into the attractor basin early and remain there for the rest of the generated trajectory.

\begin{figure}[hbtp]
    \centering
    \includegraphics[width=0.8\textwidth]{a03_results/histogram_success_ratio.png}
    \caption{Distribution of successfully generated files in the range of target attractor ratios.}
    \label{fig:hist_success}
\end{figure}

\subsection{reconstruction results}
\todo{Maciej}
czego sie dowiedzielismy.
Tutaj musi się pojawić też coś, co usprawiedliwi to, że nie braliśmy scoringu BDE

\section{conclusions.}
\todo{Wszyscy}
przede wszystkim nasza konkluzja co do guideline'u jakie parametry powinno sie stosowac w zaleznosci od specyfiki naszej sieci

\begin{center}
    \huge \textbf{PART II: Biological Model Validation} 
\end{center}

\section{introduction}
\todo{Marcin}
co to za model, jaka jest jego specyfika, moze kilka slow co on biologicznie znaczy i jaka jest jego wartosc

\section{Model Selection}
% Model Selection Pipeline

The biological validation phase required selecting real Boolean network models from established repositories. We developed a multi-stage selection pipeline to identify suitable candidates from the biodivine-boolean-models repository, a comprehensive collection of published biological Boolean network models.

\subsection{Data Source}

The biodivine-boolean-models repository contains over 200 curated Boolean network models from the literature, representing various biological systems including cell cycle regulation, signaling pathways, and gene regulatory networks. Models are provided in the \texttt{.bnet} format, specifying Boolean update functions for each network variable.

\subsection{Selection Pipeline}

The selection process was implemented through a series of Python scripts that form the \texttt{biodivine\_importer} module:

\subsubsection{Stage 1: Import and Organization}

\begin{enumerate}
    \item \textbf{Model Import (\texttt{importer.py}):} All \texttt{model.bnet} files were extracted from the source repository and renamed to a standardized format (\texttt{model\_XXX.bnet}) based on their unique identifiers.
    \item \textbf{Dimension-based Organization (\texttt{organize\_by\_dimension.py}):} Models were organized into subdirectories based on their dimensionality (number of variables). This yielded models ranging from 4D to over 800D, with the majority concentrated in the 5--50 variable range.
\end{enumerate}

\subsubsection{Stage 2: Compatibility Filtering}

\begin{enumerate}
    \item \textbf{External Input Detection (\texttt{count\_inputs.py}):} Many biological models include external input variables (environmental signals, experimental conditions) that are referenced but not defined within the network. Models with external inputs require special handling---either the inputs must be substituted with constant values, or multiple network variants must be generated for different input combinations.
    \item \textbf{Dimension Filtering:} Based on the results from Part I, only models with dimensions between 4 and 16 were retained, as higher-dimensional networks exhibited poor reconstruction quality regardless of parameter tuning.
\end{enumerate}

\subsubsection{Stage 3: Format Conversion}

The \texttt{convert\_to\_functions.py} script performed the following transformations:
\begin{itemize}
    \item \textbf{Variable Normalization:} Original variable names (e.g., \texttt{v\_CcrM}, \texttt{v\_CtrA}) were mapped to a standardized format (\texttt{x1}, \texttt{x2}, ...) while preserving the original mapping in documentation.
    \item \textbf{External Input Substitution:} For models with external inputs, multiple variants were generated by substituting all possible combinations of constant values (0 or 1) for each external input.
    \item \textbf{Transition Graph Generation:} Asynchronous state transitions were computed from the Boolean functions, consistent with the synthetic network format.
\end{itemize}

\subsection{Final Model Selection}

From the filtered and converted models, three networks were selected for the biological validation experiments:

\begin{enumerate}
    \item \textbf{bn\_109\_5d.py (5D):} A model of the \textit{Caulobacter crescentus} cell cycle regulatory network, featuring 5 key regulators: CcrM, CtrA, DnaA, GcrA, and SciP. This compact model has no external inputs and represents a well-studied bacterial cell cycle.
    
    \item \textbf{bn\_158\_7d.py (7D):} A model of the bacteriophage $\lambda$ lysis-lysogeny decision circuit, with 7 variables including CI, Cro, and N proteins. This classic system demonstrates bistable genetic switching behavior.
    
    \item \textbf{bn\_063\_10d\_011.py (10D):} A model of embryonic development regulatory network with 10 variables. This variant was generated by substituting external inputs (\texttt{v\_Ge}=0, \texttt{v\_Le}=1, \texttt{v\_Lem}=1) with specific constant values, representing a particular developmental context.
\end{enumerate}

\subsection{Selection Rationale}

The three selected models were chosen to:
\begin{itemize}
    \item \textbf{Span the tractable dimension range:} 5D, 7D, and 10D represent low, medium, and borderline-high complexity based on Part I results.
    \item \textbf{Include biological diversity:} The models represent different biological contexts (bacterial cell cycle, phage decision, embryonic development).
    \item \textbf{Minimize external dependencies:} Priority was given to models with few or no external inputs to ensure self-contained network dynamics.
\end{itemize}

\subsection{Structural Properties of Selected Models}

Table~\ref{tab:biodivine_structure} summarizes the structural properties of the three selected biological networks. Table~\ref{tab:biodivine_scc_sizes} shows the attractor and transient SCC size distributions for each model.

\begin{figure}[hbtp]
    \centering
    \includegraphics[width=0.6\textwidth]{b02_model_selection/bn_109_5d_scc_simplified.png}
    \caption{SCC graph of the Caulobacter cell cycle network (5D). The single fixed-point attractor is shown in green, with 7 transient SCCs forming the basin of attraction. The largest transient SCC contains 25 states.}
    \label{fig:scc_5d}
\end{figure}

\begin{figure}[hbtp]
    \centering
    \includegraphics[width=0.8\textwidth]{b02_model_selection/bn_158_7d_scc_simplified.png}
    \caption{Simplified SCC graph of the phage lambda decision network (7D). Two attractors are visible: a fixed point and a 2-state cycle representing the bistable lysis-lysogeny switch. The 79 transient SCCs reflect the complex decision-making landscape.}
    \label{fig:scc_7d}
\end{figure}

\begin{figure}[hbtp]
    \centering
    \includegraphics[width=0.8\textwidth]{b02_model_selection/bn_063_10d_011_scc_simplified.png}
    \caption{Simplified SCC graph of the embryonic development network (10D). Despite having 704 transient SCCs, the network converges to a single fixed-point attractor. The two largest transient SCCs contain 221 and 100 states respectively.}
    \label{fig:scc_10d}
\end{figure}

\begin{table}[h!]
    \centering
    \caption{Structural properties of selected biological networks}
    \label{tab:biodivine_structure}
    \begin{tabular}{clccccc}
        \hline
        \textbf{Dim} & \textbf{Parentless} & \textbf{Fixed Pts} & \textbf{Attractors} & \textbf{Trans. SCCs} & \textbf{SCC Edges} \\ \hline
        5D   & 1 & 1 & 1 & 7   & 13   \\
        7D     & 1 & 1 & 2 & 79  & 294  \\
        10D     & 1 & 1 & 1 & 704 & 3723 \\ \hline
    \end{tabular}
\end{table}

\begin{table}[h!]
    \centering
    \caption{Attractor and transient SCC sizes for biological networks}
    \label{tab:biodivine_scc_sizes}
    \begin{tabular}{clll}
        \hline
        \textbf{Dim} & \textbf{Description} & \textbf{Attractor Sizes} & \textbf{Transient SCC Sizes (top 5)} \\ \hline
        5  & Caulobacter cell cycle   & [1]      & [25, 1, 1, 1, 1]      \\
        7  & Phage lambda decision    & [2, 1]   & [31, 11, 4, 2, 2]     \\
        10 & Embryonic development    & [1]      & [221, 100, 1, 1, 1]   \\ \hline
    \end{tabular}
\end{table}

All three networks exhibit fixed-point attractors, with the phage lambda model additionally containing a 2-state cyclic attractor representing the bistable lysis-lysogeny switch. The 10D embryonic development network has a notably large number of transient SCCs (704), reflecting the complex developmental trajectory landscape.

\subsection{SCC Graph Visualizations}

To better understand the structure of the selected networks, we generated condensation graphs showing the relationships between strongly connected components. In these visualizations, green nodes represent attractors (terminal SCCs), blue nodes represent transient SCCs, and node labels indicate the number of states in each SCC.


\section{methodology}
\todo{Mateusz}
jakie parametry trajektorii/rekonstrukcji wybrano i dlaczego takie, w odniesieniu do konkluzji czesci pierwszej

\section{results}
\todo{Wszyscy}
jak dobrze poszla rekonstrukcja, czy zgodnie z przewidywaniami, czy podobnie jak wychodzilo w sieciach syntetycznych

\section{conclusions / closing remarks}
\todo{Marcin}
krotko, jak sobie poradzil pipeline w empirycznej sieci. moze krotko co oznaczalaby zrekonstruowana siec biologicznie np. wg zrekonstruowanej sieci jak dzialaby szlak bialkowy etc.

\end{document}
